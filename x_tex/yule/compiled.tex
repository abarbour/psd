\documentclass[preprint,authoryear,12pt]{elsarticle}
\usepackage{amssymb}
\journal{Computers and Geosciences}
\begin{document}
\begin{frontmatter}
\title{Reducing the potential for spectral bias in 
rlpSpec}
\author{Andrew J Barbour, Robert L Parker}
\address{}
\begin{abstract}
Bias in power spectral density estimates 
of geophysical datasets 
may be significantly reduced by removing
outliers from the dataset under consideration.  
In many such datasets,
outliers with relatively large amplitudes
may be easily identified, but datasets can often have
numerous outliers which are too small to be identified visually.
We show that by modeling the statistical behavior of such
data as an auto-regressive (AR)
process, small outliers may easily be identified and removed; 
consequently, the estimated spectrum more accurately reflects
the true frequency content of the signal analyzed.
This is demonstrated using a few sample datasets traditionally used in the
geophysics community.
\end{abstract}
\begin{keyword}
\end{keyword}
\end{frontmatter}
% yule
% Fri Jan 27 12:53:41 PST 2012
% /Users/abarbour/nute.processing/development/rlpSpec/tex/yule
%
%\newcommand{\new}[#in]{code}
%\renewcommand{\new}[#in]{code}

%biblio
\bibliography{yule_link.bib}
\bibliographystyle{/Users/abarbour/texsty/agu/agu}

\end{document}
