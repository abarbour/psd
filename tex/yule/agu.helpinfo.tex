%%%%%%%%%%%%%%%%%%%%%%%%%%%%%%%%%%%%%%%%%%%%%%%%%%%%%%%%%%%%%%%%%%%%%%%%%%%%
% AGUtmpl.tex: this template file is for articles formatted with LaTeX2e,
% Modified March 2009
%
% This template includes commands and instructions
% given in the order necessary to produce a final output that will
% satisfy AGU requirements.
%
% PLEASE DO NOT USE YOUR OWN MACROS
%
% For more information on using the AGUTeX macro package,
% see agudocs.tex or agudocs.pdf
%
%%%%%%%%%%%%%%%%%%%%%%%%%%%%%%%%%%%%%%%%%%%%%%%%%%%%%%%%%%%%%%%%%%%%%%%%%%%%
%
% All questions should be e-mailed to author.help@agu.org.
%
%%%%%%%%%%%%%%%%%%%%%%%%%%%%%%%%%%%%%%%%%%%%%%%%%%%%%%%%%%%%%%%%%%%%%%%%%%%%
%
% Step 1: set the \documentclass
%
% The three options for article format are: two-column (default),
% draft, for initial article submission; and galley for narrow
% single columns.
%
% PLEASE USE THE DRAFT OPTION TO SUBMIT YOUR PAPERS
% The draft option produces double spaced output
%
% Choose the journal abbreviation for the journal you are
% submitting to:
% jgrga JOURNAL OF GEOPHYSICAL RESEARCH
% gbc   GLOBAL BIOCHEMICAL CYCLES
% grl   GEOPHYSICAL RESEARCH LETTERS
% pal   PALEOCEANOGRAPHY
% ras   RADIO SCIENCE
% rog   REVIEWS OF GEOPHYSICS
% tec   TECTONICS
% wrr   WATER RESOURCES RESEARCH
% gc    GEOCHEMISTRY, GEOPHYSICS, GEOSYSTEMS
% (If you are submitting to a journal other than jgrga,
% substitute the initials of the journal for "jgrga" below)
%%%%%%%%%%%%%%%%%%%%%%%%%%%%%%%%%%%%%%%%%%%%%%%%%%%%%%%%%%
%%%% optional article formats author might want to use
% To produce a galley version:
% \documentclass[galley,jgrga]{AGUTeX}
% To produce a two columned version:
% \documentclass[jgrga]{AGUTeX}
%%%%%%%%%%%%%%%%%%%%%%%%%%%%%%%%%%%%%%%%%%%%%%%%%%%%%%%%%%%%%%%%%%%%%%%%%
% OPTIONAL:
% To print your article using PostScript fonts, uncomment this:
% \usepackage{agu-ps}
% You many need to edit the top of agu-ps to use the names of the PS
% fonts on your system.
%%%%%%%%%%%%%%%%%%%%%%%%%%%%%%%%%%%%%%%%%%%%%%%%%%%%%%%%%%%%%%%%%%%%%%%%%
% OPTIONAL:
% To Create numbered lines:
% If you don't already have lineno.sty, you can download it from
% http://www.ctan.org/tex-archive/macros/latex/contrib/ednotes/
% (or google lineno.sty ctan), available at TeX Archive Network (CTAN).
% Take care that you always use the latest version.
% To activate the commands, uncomment \usepackage{lineno}
% and \linenumbers*[1]command, below:
%%%%%%%%%%%%%%%%%%%%%%%%%%%%%%%%%%%%%%%%%%%%%%%%%%%%%%%%%%%%%%%%%%%%%%%%%
% Figures and Tables
%
% When submitting articles through the GEMS system:
% COMMENT OUT ANY COMMANDS THAT INCLUDE GRAPHICS.
% (See FIGURES section near the end of the file)
%  Figures and Tables should be placed at the end of the article,
%  after the references.
%
%  Uncomment the following command to include .eps files
%  (comment out this line for draft format):
%
%    Uncomment the following command to allow illustrations to print
%    when using Draft:
%  \setkeys{Gin}{draft=false}
%
% Substitute one of the following for [dvips] above
% if you are using a different driver program and want to
% proof your illustrations on your machine:
%
% [xdvi], [dvipdf], [dvipsone], [dviwindo], [emtex], [dviwin],
% [pctexps],  [pctexwin],  [pctexhp],  [pctex32], [truetex], [tcidvi],
% [oztex], [textures]
%
% See how to enter figures and tables at the end of the article, after
% references.
%
%% ------------------------------------------------------------------------ %%
%
%  ENTER PREAMBLE
%
%% ------------------------------------------------------------------------ %%
% Author names in capital letters:
% Shorter version of title entered in capital letters:
% Author mailing address: please repeat this command for
% each author and alphabetize authors:
%\authoraddr{J. R. McConnell, Division of Hydrologic
%Sciences, 123 Main Street, Desert Research Institute, Reno, NV
%89512, USA.}
%\authoraddr{E. Mosley-Thompson, Department of Geography,
%Ohio State University, 123 Orange Boulevard, Columbus, OH 43210,
%USA.}
%\authoraddr{R. Williams, Department of Space Sciences, University of
%Michigan, 123 Brown Avenue, Ann Arbor, MI 48109, USA.}
%% ------------------------------------------------------------------------ %%
%
%  TITLE
%
%% ------------------------------------------------------------------------ %%
%
% e.g., \title{Terrestrial ring current:
% Origin, formation, and decay $\alpha\beta\Gamma\Delta$}
% You may use \\ to break the title over several lines.
%% ------------------------------------------------------------------------ %%
%
%  AUTHORS AND AFFILIATIONS
%
%% ------------------------------------------------------------------------ %%
%Use \author{\altaffilmark{}} and \altaffiltext{}
% \altaffilmark will produce footnote;
% matching altaffiltext will appear at bottom of page.
% May use \\ to start a new line.
%E. Mosley-Thompson, \altaffilmark{2} R. Williams, \altaffilmark{3}
%and J. R. McConnell\altaffilmark{4}}
%\altaffiltext{2}{Department of Geography, Ohio State University,
%Columbus, Ohio, USA.}
%\altaffiltext{3}{Department of Space Sciences, University of
%Michigan, Ann Arbor, Michigan, USA.}
%\altaffiltext{4}{Division of Hydrologic Sciences, Desert Research
%Institute, Reno, Nevada, USA.}
%% ------------------------------------------------------------------------ %%
%
%  ABSTRACT
%
%% ------------------------------------------------------------------------ %%
% >> Do NOT include any \begin...\end commands within
% >> the body of the abstract.
%%% End of body of article:
%%%%%%%%%%%%%%%%%%%%%%%%%%%%%%%%
%% Optional Appendix goes here
%
%%%%%%%%%%%%%%%%%
% Geophysical Research Letters only allows an appendix without a letter.
%% You can get this result with
%  \section*{Appendix}
%  or
%  \section*{Appendix: Title}
%%%%%%%%%%%%%%%%%
%
% \appendix resets counters and redefines section heads
% but doesn't print anything.
% After typing  \appendix
%
% \section{Here Is Appendix Title}
% will print
% Appendix A: Here Is Appendix Title
%
% \section*{Appendix}
% will print
% Appendix
%
% \section*{Appendix: Here Is Appendix Title}
% will print
% Appendix: Here Is Appendix Title
%
% For only 1 appendix \appendix \section{Appendix} is preferred.
% which will print
% Appendix A
%%%%%%%%%%%%%%%%%%%%%%%%%%%%%%%%%%%%%%%%%%%%%%%%%%%%%%%%%%%%%%%%
%
% Optional Glossary or Notation section, goes here
%
%%%%%%%%%%%%%%
%% ------------------------------------------------------------------------ %%
%
%  REFERENCE LIST AND TEXT CITATIONS
%
% Either type in your references using
% \begin{thebibliography}{}
% \bibitem{}
% Text
% \end{thebibliography}
%
% Or,
%
% If you use BiBTeX for your References, please produce your .bbl
% file and copy the contents into your paper here.
%
% Follow these steps:
% 1. Run LaTeX on your LaTeX file.
%
% 2. Run BiBTeX on your LaTeX file.
%
% 3. Open the new .bbl file containing the reference list and
%   copy all the contents into your LaTeX file here.
%
% 4. Comment out the old \bibliographystyle and \bibliography commands.
%
% 5. Run LaTeX on your new file before submitting.
%
% AGU does not want a .bib or a .bbl file, but asks that you
% copy in the contents of your .bbl file here.
%\bibitem[{\textit{Kilby}(2008)}]{jskilby}
%Kilby, J. S. (2008), Invention of the integrated circuit, {\it IEEE
%Trans. Electron Devices,} \textit{23}, 648--650.
%\bibitem[{\textit{Kilby et al.}(2008)}]{jskilbye}
%Kilby, J. S., S. Smith, and R. Jones (2008), Invention of the
%integrated circuit, {\it IEEE Trans. Electron Devices,} \textit{23},
%648--650.
%Reference citation examples:
%...as shown by \textit{Kilby} [2008].
%...has been shown [\textit{Kilby et al.}, 2008].
%...as shown by \cite{jskilby}.
%...has been shown \citep{jskilbye}.
%% ------------------------------------------------------------------------ %%
%
%  END ARTICLE
%
%% ------------------------------------------------------------------------ %%
%% Enter Figures and Tables here:
% When submitting articles through the GEMS system:
% COMMENT OUT ANY COMMANDS THAT INCLUDE GRAPHICS.
% Figure captions go below this illustration; Table captions go above tables
% ONE-COLUMN figure/table, including eps graphics
%
% \begin{figure}
% \noindent\includegraphics[width=20pc]{samplefigure.eps}
% \caption{Caption text here}
% \end{figure}
% \end{document}
%
% \begin{table}
% \caption{}
% \end{table}
%
% ---------------
% TWO-COLUMN figure/table
%
% \begin{figure*}
% \noindent\includegraphics[width=39pc]{samplefigure.eps}
% \caption{Caption text here}
% \end{figure*}
%
% \begin{table*}
% \caption{Caption text here}
% \end{table*}
%
% see below for how to make landscape figures or tables
%%% End the article here:
%%%%%%%%%%%%%%%%%%%%%%%%%%%%%%%%%%%%%%%%%%%%%%%%%%%%%%%%%%%%%%%
%  SECTION HEADS
%% ------------------------------------------------------------------------ %%
%
%  IN-TEXT LISTS
%
%% ------------------------------------------------------------------------ %%
% Do not use bulleted lists; enumerated lists are okay.
%% ------------------------------------------------------------------------ %%
%
%  EQUATIONS
%
%% ------------------------------------------------------------------------ %%
% Single-line equations are centered.
% IF YOU HAVE MULTI-LINE EQUATIONS, PLEASE
% BREAK THE EQUATIONS INTO TWO OR MORE LINES
% OF SINGLE COLUMN WIDTH (20 pc, 8.3 cm)
% using double backslashes (\\).
% To create multiline equations, use the
% \begin{eqnarray} and \end{eqnarray} environment
% as demonstrated below.
% Break each line at a sign of operation
% (+, -, etc.) if possible, with the sign of operation
% on the new line.
% Indent second and subsequent lines to align with
% the first character following the equal sign on the
% first line.
% Use an \hspace{} command to insert horizontal space
% into your equation if necessary. Place an appropriate
% unit of measure between the curly braces, e.g.
% \hspace{1in}; you may have to experiment to achieve
% the correct amount of space.
% There is another multiline equation environment:
% \begin{aguleftmath}...\end{aguleftmath}
% The equation is aligned left and the second line indents to
% the width of a paragraph indent (AGU style)
%% ------------------------------------------------------------------------ %%
%
%  EQUATION NUMBERING: COUNTER
%
%% ------------------------------------------------------------------------ %%
% You may change equation numbering by resetting
% the equation counter or by explicitly numbering
% an equation.
% To explicitly number an equation, type \eqnum{}
% (with the desired number between the brackets)
% after the \begin{equation} or \begin{eqnarray}
% command.  The \eqnum{} command will affect only
% the equation it appears with; LaTeX will number
% any equations appearing later in the manuscript
% according to the equation counter.
%
% To reset the equation counter, place the setcounter{equation}
% command in front of your equation(s).
%\setcounter{equation}{0}
% Set the equation counter to 0 if the next
% number needed is 1 or set it to 7 if the
% next number needed is 8, etc.
%
% The \setcounter{equation} command does affect
% equations appearing later in the manuscript.
% If you have a multiline equation that needs only
% one equation number, use a \nonumber command in
% front of the double backslashes (\\) as shown in
% the multiline equation above.
%%%%%%%%%%%%%%%%%%%%%%%%%%%%%%%%%%%%%%%%%%%%%%%%%%%%%%
%% Landscape figure and table examples
%
% ---------------
% Landscape (broadside) figure/table
% (These objects will not display properly in draft mode, use galley.)
%
% ONE-COLUMN landscape figure and table
%
% \begin{landscapefigure}
% \includegraphics[height=.75\mycolumnwidth,width=42pc]{samplefigure.eps}
% \caption{Caption text here}
% \end{landscapefigure}
%
% \begin{landscapetable}
% \caption{Caption text here}
% \begin{tabular*}{\hsize}{@{\extracolsep{\fill}}lcccc}
% \tableline
% ....
% \tableline\\
% \multicolumn5l{(a) Algorithms from Numerical Recipes}\\
% \end{tabular*}
% \tablenotetext{}{}
% \tablecomments{}
% \end{landscapetable}
%
% FULL-PAGE landscape figures and tables
%
% \begin{figure*}[p]
% \begin{landscapefigure*}
% illustration here
% \caption{caption here}
% \end{landscapefigure*}
% \end{figure*}
%
% \begin{table}[p]
% \begin{landscapetable*}
% \caption{}
% \begin{tabular*}{\textheight}{@{\extracolsep{\fill}}lccrrrcrrr}
% ....
% \end{tabular*}
% \begin{tablenotes}
% ...
% \end{tablenotes}
% \end{landscapetable*}
% \end{table}
%
%\end{document}
