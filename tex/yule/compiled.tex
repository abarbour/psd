\documentclass[draft,jgrga]{agutex}
%\usepackage{lineno}
%\linenumbers*[1]
\usepackage[dvips]{graphicx}
\authorrunninghead{BARBOUR ET AL.}
\titlerunninghead{SHORT TITLE}
\authoraddr{A. J. Barbour,
9500 Gilman Dr. , La Jolla, Ca 92093-0208, USA.
(abarbour@ucsd.edu)}
\begin{document}
\title{\input{title}}
\authors{Andrew J. Barbour, \altaffilmark{1}}
\altaffiltext{1}{Institute of Geophysics and Planetary Physics,
Scripps Insitution of Oceanography,
La Jolla, California, USA.}
\begin{abstract}
\input{abstract}
\end{abstract}
\begin{article}
% yule
% Fri Jan 27 12:53:41 PST 2012
% /Users/abarbour/nute.processing/development/rlpSpec/tex/yule
%
%\newcommand{\new}[#in]{code}
%\renewcommand{\new}[#in]{code}

%biblio
\bibliography{yule_link.bib}
\bibliographystyle{/Users/abarbour/texsty/agu/agu}

\section{Introduction}
(Article text here)
cite \cite{2008o}
\begin{thebibliography}{}
\bibitem[{\textit{Brenguier et~al.}(2008)\textit{Brenguier, Campillo,
  Hadziioannou, Shapiro, Nadeau, and Larose}}]{2008o}
Brenguier, F., M.~Campillo, C.~Hadziioannou, N.~M. Shapiro, R.~M. Nadeau, and
  E.~Larose (2008), Postseismic relaxation along the san andreas fault at
  parkfield from continuous seismological observations, \textit{Science},
  \textit{321}, 1478.
\end{thebibliography}
\end{article}
\end{document}
More Information and Advice:
 ---------------
 Level 1 head
 Use the \section{} command to identify level 1 heads;
 type the appropriate head wording between the curly
 brackets, as shown below.
 Capitalize the first letter of each word (expect for
 prepositions, conjunctions, and articles that are
 three or fewer letters).
 Do not hyphenate level 1 heads. To break lines,
 type \protect\\ where you want the break to occur.
 AGU prefers the inverted triangle, breaking before
 prepositions, conjunctions, and articles, if possible.
An example:
\section{Level 1 Head: Introduction}
 ---------------
 Level 2 head
 Use the \subsection{} command to identify level 2 heads.
 Capitalize the first letter of each word (expect for
 prepositions, conjunctions, and articles that are
 three or fewer letters).
 Do not hyphenate level 1 heads. To break lines,
 type \protect\\ where you want the break to occur.
 AGU prefers the inverted triangle, breaking before
 prepositions, conjunctions, and articles, if possible.
\subsection{Level 2 Head} An example.
 ---------------
 Level 3 head
 Use the \subsubsection{} command to identify level 3 heads
 Capitalize only the first letter of the first word, acronyms,
 first letter of proper nouns, and first letter of first word
 after a colon.
 Hyphenation is permitted in level 3 heads, if needed.
\subsubsection{Level 3 Head} An example.
\subsubsubsection{Level 4 Head} An example.
 \begin{enumerate}
 \item
 \item
 \item
 \end{enumerate}
 Math coded inside display math mode \[ ...\]
 will not be numbered e.g.:
 \[ x^2=y^2 + z^2\]
 Math coded inside \begin{equation} and \end{equation} will
 be automatically numbered e.g.:
 \begin{equation}
 x^2=y^2 + z^2
 \end{equation}
\begin{eqnarray}
  x_{1} & = & (x - x_{0}) \cos \Theta \nonumber \\
        && + (y - y_{0}) \sin \Theta  \nonumber \\
  y_{1} & = & -(x - x_{0}) \sin \Theta \nonumber \\
        && + (y - y_{0}) \cos \Theta.
\end{eqnarray}
If you don't want an equation number, use the star form:
\begin{eqnarray*}...\end{eqnarray*}
