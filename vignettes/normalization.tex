\documentclass[12pt]{article}\usepackage{graphicx, color}
%% maxwidth is the original width if it is less than linewidth
%% otherwise use linewidth (to make sure the graphics do not exceed the margin)
\makeatletter
\def\maxwidth{ %
  \ifdim\Gin@nat@width>\linewidth
    \linewidth
  \else
    \Gin@nat@width
  \fi
}
\makeatother

\IfFileExists{upquote.sty}{\usepackage{upquote}}{}
\definecolor{fgcolor}{rgb}{0.2, 0.2, 0.2}
\newcommand{\hlnumber}[1]{\textcolor[rgb]{0,0,0}{#1}}%
\newcommand{\hlfunctioncall}[1]{\textcolor[rgb]{0.501960784313725,0,0.329411764705882}{\textbf{#1}}}%
\newcommand{\hlstring}[1]{\textcolor[rgb]{0.6,0.6,1}{#1}}%
\newcommand{\hlkeyword}[1]{\textcolor[rgb]{0,0,0}{\textbf{#1}}}%
\newcommand{\hlargument}[1]{\textcolor[rgb]{0.690196078431373,0.250980392156863,0.0196078431372549}{#1}}%
\newcommand{\hlcomment}[1]{\textcolor[rgb]{0.180392156862745,0.6,0.341176470588235}{#1}}%
\newcommand{\hlroxygencomment}[1]{\textcolor[rgb]{0.43921568627451,0.47843137254902,0.701960784313725}{#1}}%
\newcommand{\hlformalargs}[1]{\textcolor[rgb]{0.690196078431373,0.250980392156863,0.0196078431372549}{#1}}%
\newcommand{\hleqformalargs}[1]{\textcolor[rgb]{0.690196078431373,0.250980392156863,0.0196078431372549}{#1}}%
\newcommand{\hlassignement}[1]{\textcolor[rgb]{0,0,0}{\textbf{#1}}}%
\newcommand{\hlpackage}[1]{\textcolor[rgb]{0.588235294117647,0.709803921568627,0.145098039215686}{#1}}%
\newcommand{\hlslot}[1]{\textit{#1}}%
\newcommand{\hlsymbol}[1]{\textcolor[rgb]{0,0,0}{#1}}%
\newcommand{\hlprompt}[1]{\textcolor[rgb]{0.2,0.2,0.2}{#1}}%

\usepackage{framed}
\makeatletter
\newenvironment{kframe}{%
 \def\at@end@of@kframe{}%
 \ifinner\ifhmode%
  \def\at@end@of@kframe{\end{minipage}}%
  \begin{minipage}{\columnwidth}%
 \fi\fi%
 \def\FrameCommand##1{\hskip\@totalleftmargin \hskip-\fboxsep
 \colorbox{shadecolor}{##1}\hskip-\fboxsep
     % There is no \\@totalrightmargin, so:
     \hskip-\linewidth \hskip-\@totalleftmargin \hskip\columnwidth}%
 \MakeFramed {\advance\hsize-\width
   \@totalleftmargin\z@ \linewidth\hsize
   \@setminipage}}%
 {\par\unskip\endMakeFramed%
 \at@end@of@kframe}
\makeatother

\definecolor{shadecolor}{rgb}{.97, .97, .97}
\definecolor{messagecolor}{rgb}{0, 0, 0}
\definecolor{warningcolor}{rgb}{1, 0, 1}
\definecolor{errorcolor}{rgb}{1, 0, 0}
\newenvironment{knitrout}{}{} % an empty environment to be redefined in TeX

\usepackage{alltt}
% !Rnw weave = knitr
% sweave fig help:
% http://users.stat.umn.edu/~geyer/Sweave/foo.pdf
% borrowing design from roxygen -- AJB Jan 13

%% \VignetteIndexEntry{Normalization of power spectral densities.}
%% \VignetteEngine{knitr}

\usepackage[utf8]{inputenc}
\usepackage{fancyvrb}
\usepackage[pdfborder={0 0 0}]{hyperref}
\usepackage{url}
\usepackage{upquote}
\usepackage{graphicx}
\usepackage{grffile}
\usepackage{amsmath}
\usepackage{amssymb}
\usepackage{float}
\usepackage{natbib}
\usepackage{geometry}
\geometry{verbose,tmargin=3cm,bmargin=5cm,lmargin=2.5cm,rmargin=2.5cm}
%%\usepackage{fullpage}
\usepackage{makeidx} % for indexing
%\usepackage{showidx} % shows page (remove when complete)
\makeindex % comment to have no index
%%
%% Mathematics definitions
%% do not use special characters!
%%
%TMP\newcommand{\dxx}[1]{\enm{#1}} %% null
\newcommand{\enm}[1]{\ensuremath{#1}} %% null
\newcommand{\dFT}[0]{\enm{\mathfrak{F}}} %% fourier transform operator (FT)
\newcommand{\dFTp}[1]{\enm{\dFT{}\{#1\}}} %% FT{...}
\newcommand{\df}[0]{\enm{x}} %% stationary signal f
\newcommand{\dF}[0]{\enm{X}} %% FT{f}
\newcommand{\dmod}[1]{\enm{\mathrm{mod} \{#1\}}}
\newcommand{\dmodF}[0]{\dmod{\dF{}}}
\newcommand{\dargF}[0]{\enm{\mathrm{arg} \{\dF{}\}}}
\newcommand{\ds}[0]{\enm{s}} %% spectrum
\newcommand{\dS}[0]{\enm{S}} %% Spectrum
\newcommand{\dsupsf}[1]{\enm{^{#1}\ds{}}} %% superscript spectrum front
\newcommand{\dsupSf}[1]{\enm{^{#1}\dS{}}} %% superscript Spectrum front
\newcommand{\dsupsb}[1]{\enm{\ds{}^{#1}}} %% superscript spectrum back
\newcommand{\dsupSb}[1]{\enm{\dS{}^{#1}}} %% superscript Spectrum back
\newcommand{\daS}[1]{\dsupsf{\mathrm{(A)}}} %% amplitude Spectrum
\newcommand{\dpS}[1]{\dsupsf{\mathrm{(\phi)}}} %% phase Spectrum
\newcommand{\dESD}[1]{\dsupSf{\mathrm{(E)}}} %% energy Spectral density
\newcommand{\dPSD}[1]{\dsupSf{\mathrm{(P)}}} %% power Spectral density
%%
\newcommand{\intone}{\enm{\int_{- \infty} ^{\infty}}}
\newcommand{\Fo}{\enm{\mathcal{F}}}
\newcommand{\Xfo}{\enm{\tilde{X}}}
\newcommand{\rXfo}{\enm{\tilde{X}^R}}
\newcommand{\iXfo}{\enm{\tilde{X}^I}}
\newcommand{\Ex}{\enm{\mathcal{E}}}
\newcommand{\slfrac}[2]{\left.#1\middle/#2\right.}
\newcommand{\half}{\enm{\slfrac{1}{2}}}
\newcommand{\Var}{\enm{\mathcal{V}}}
\newcommand{\conv}{\ensuremath{\ast}}
%%

%%
\newcommand{\SC}[1]{\textsc{#1}}
\newcommand{\SCY}[0]{\SC{Yes}}
\newcommand{\SCN}[0]{\SC{No}}
\newcommand{\Rcmd}[1]{\texttt{#1}}
\newcommand{\rlp}[0]{\Rcmd{rlpSpec}}
\newcommand{\naive}[0]{na\"{\i}ve}
\newcommand{\bidx}[1]{\index{#1}{\textbf{#1}}} 
\newcommand{\idx}[1]{\index{#1}{#1}} 
%% path, filename, caption, label
\newcommand{\listing}[4]{        %
  \begin{figure}[H]              %
    \centering                   %
    \VerbatimInput[numbers=left, %
      frame=single,              %
      label=#2]{#1}              %
    \caption{#3}                 %
    \label{#4}                   %
  \end{figure}                   %
}
\author{Andrew J. Barbour}
\title{Normalization of Power Spectral Densities.}
\begin{document}
\maketitle
\begin{abstract}
  A vast and deep pool of
  literature exists on the subject of spectral 
  analysis; wading through it can
  obscure even the most
  fundamental concepts
  to the inexperienced practitioner.
  Appropriate interpretation of spectral analyses
  depends crucially on the normalization used, and
  here we outline
  the normalization used by \rlp{}, namely
  the \bidx{single-sided}
  \bidx{power spectral density} (PSD).
  We briefly outline the background mathematics,
  present an example from scratch,
  and compare the results with
  the normalization used by 
  the spectrum estimator included in the base distribution of
  R: \Rcmd{stats::spectrum}.
\end{abstract}
\tableofcontents
\section{Background}
There can often be confusion about the different quantities used
in spectral analysis\footnote{
A nice illustration of the type of
confusion common in spectral analyses
of confusion is found in
this thread on \Rcmd{R-help}:\\
\url{http://r.789695.n4.nabble.com/Re-How-do-I-normalize-a-PSD-td792902.html}
},
partly due to myriad nomenclature within the incredibly vast literature
on the subject.
Commonly one finds similarly sounding phrases, including ``amplitude spectrum",
``energy spectral density", ``power", ``power spectra", and even "spectra".
These all mean \emph{something}, but are rarely equivalent, and can be used improperly.

Let us, for the sake of brevity, assume we are in 
the time domain, and we are considering a discrete stationary signal \df{}
of length $N$,
having a Discrete Fourier Transform \dFTp{f} represented by\dF{}.
The \bidx{amplitude spectrum}
of this transform pair is simply the 
amplitude of \dF{}, or \dmodF{}, which we will denote this as \daS{}.
This transform pair's corresponding \bidx{phase spectrum}
is the phase angle of \dF{}, or \dargF{}, denoted by \dpS{}.

How do we interpret the quantities 

Table with equivalent expressions
% needs
%\newcommand{\bidx}[1]{\index{#1}{\textbf{#1}}} 
%\newcommand{\idx}[1]{\index{#1}{#1}} 
%\newcommand{\FT}[0]{\mathfrak{F}}
%\newcommand{\FTp}[1]{\FT{}\{#1\}}
%\newcommand{\rlp}[0]{\Rcmd{rlpSpec}}
%\newcommand{\SC}[1]{\textsc{#1}}
%\newcommand{\SCY}[0]{\SC{Yes}}
%\newcommand{\SCN}[0]{\SC{No}}
\begin{table}[htbp!]
\centering
\caption{Representors of various spectral quantities.}
\begin{tabular}{r l l l l}
\hline
\SC{Expression} & \SC{Representing} & \SC{Equivalent expression} & & \\
\hline
\df{} & stationary timeseries &  &  &  \\
\dF{} & Fourier transform of \df{} & \dFTp{f} &  &  \\
\daS{} & \bidx{amplitude spectrum} of \dF{} & $|\dF{}|$ or  \dmodF{} & & \\
\dpS{} & \bidx{phase spectrum} of \dF{} & \dargF{} & & \\
\hline
\end{tabular}
\label{tbl:norm}
\end{table}


\section{DCA}
To fix the first problem we define $X_T (t)$ as the process 
$X (t)$ on the finite interval $(-T/2, \,T/2)$:
%
\[
X_T (t) = \left \{
\begin{array}{cc}
X(t) & -T/2 \le t \le T/2\\
0 & \mathrm{ otherwise.} \\
\end{array}
\right .
\]
%
This is not stationary, but that is all right for now.
Any particular realization of this process
has a bounded 2-norm and thus has an ordinary Fourier transform:
%
\[
\Xfo_T (f) = \Fo [ X_T (t)] = 
\intone X_T (t) e^{-2 \pi i ft} \, dt = 
\int_{-T/2}^{T/2}  X (t) e^{-2 \pi i ft} \, dt
\]
%
But $\Xfo_T$ is still a random function of $f$;
to remove this,
we we find its squared magnitude and take the expected value:
$\Ex [ | \Xfo_T (f) | ^ 2 ]$.

We then let $T$ tend to infinity;
but we can easily see that if we do this,
this expected value would grow to infinity -- and
so we divide it by the interval length $2T$ to tame the growth.
Putting all this together gives us a function of frequency
%
\begin{equation}
\label{eq-psddef1}
\begin{split}
S(f) & = \lim_{T \to \infty} \frac{1}{T}
\Ex [ | \Xfo_T (f) | ^ 2 ] \\
& = \lim_{T \to \infty} \Ex \left [ \frac{1}{T} \left | 
\int_{-T/2}^{T/2}  X (t) e^{-2 \pi i ft} \, dt \right | ^ 2 \right ]
\end{split}
\end{equation}

Equation (\ref{eq-psddef1}) defines the power spectral density;
\label{sy-psd}
\index{Power spectral density!Fourier transform definition}
it can be shown that $S(f)$ exists for all
stationary processes $X$ with zero mean and a bounded variance.
It is, obviously, real and non-negative.

\subsection{norm}
You should be aware that equation (\ref{eq-psddef1}) defines
a particular normalization for the PSD: one that is not always used,
and indeed is probably the less common one.
Equation (\ref{eq-psddef1}) defines what is called
the \textbf{two-sided} PSD,
because in it we allow $f$ to run from $-\infty$ to $\infty$.
As with the Fourier tranform of a real function,
when $X$ is real the power spectrum $S(f)$ is even:
we only need the values for $f \ge 0$.
What is used in this case is usually the
\textbf{one-sided} PSD,
\index{Power spectral density!one-sided}
which is given by $2 S(f)$ for $f \ge 0$;
we explain the factor of two below.
In using other people's spectra,
you should be aware that the spectrum,
unspecified, might be either the two-sided spectrum
shown for positive $f$ only, or the one-sided spectrum -- and
you should,
even at the risk of being accused of pedantry,
specify which one you have used.

Looking at (\ref{eq-psddef1}) we can observe that $S(f) $ at any particular $f$
is obtained from products of $X$ with itself,
so it is related only to the second order moment of $X$
-- no third order moments are involved.
We already introduced a function
involving second-order moments of $X$, namely
the autocovariance $R(\tau$,
defined in equations (\ref{eq-covarts}) and (\label{eq-covarco}).
Does $S(f)$ provide independent information about $X$,
or is there a connection between $R(\tau)$ and $S(f)$?
Somewhat surprisingly the answer is that the
functions $R(\tau)$ and $S(f)$
contain \textit{exactly} the same information,
because $S(f)$ is just the Fourier transform of $R(\tau)$
\index{Power spectral density!autocovariance definition}
%
\begin{equation}
\label{eq-psddef2}
S(f) = \Fo [ R(t) ] =
\intone R(t) e ^ {-2 \pi i ft} \, dt
\end{equation}
%
Equation (\ref{eq-psddef2}) is often used as an alternative definition
of the PSD,
not least because it is
(as we will see)
more useful than equation (\ref{eq-psddef1})
for proving theorems about the PSD.

For real data, $R(\tau)$ is not just even (as it always is)
but also real.
In that case we can write
%
\begin{equation}
\label{eq-psddef3}
S(f) = \half \int_0^{\infty} R(t) \cos(-2 \pi ft) \, dt
\end{equation}
%
where $S(f)$ is the two-sided spectrum;
the $\half$ absorbs a factor of two
produced when we change the limits of the integral.
If we used the one-sided spectrum (as makes
sense for a real series) we would omit
the $\half$;
for generality
we will use the definition given by equation (\ref{eq-psddef2}).

Before we establish the truth of (\ref{eq-psddef2}) we note a few
consequences of it.
Since $R(t)$ is a real even function of $t$, (\ref{eq-psddef2})
implies that $S(f)$ is also a real and even function in $f$.
But the fact that
$S(f)$ must be non-negative puts severe restrictions on what
functions $R(t)$ might be autocovariances; clearly
not every even function
with a Fourier transform will have a positive
Fourier transform.

The inverse transform of (\ref{eq-psddef2}) is
%
\begin{equation}
\label{eq-psdinv}
R(t) = \intone S(f) e ^ {2 \pi i ft} \, df
\end{equation}
%
and recall from the definition of $R(t)$ that
%
\[
R(0) = \Ex [X(t) X(t)] = \Var [X] = \sigma ^ 2
\]
%
remembering that $X$ is a zero-mean process.
Setting $t = 0$ in
(\ref{eq-psddef2}) gives an important result:
%
\begin{equation}
\label{eq-psdnorm}
\sigma ^ 2 = \intone S (f) \, df
\end{equation}
%
That is,
\textit{the area under the power spectrum is the variance
of the process.}
It is to preserve this property that
we double $S(f)$ if we use the one-sided PSD;
then the integrated area is again the variance even though we
we integrate $S(f)$ only over positive frequencies.

Now we verify (\ref{eq-psddef2});
the argument follows the lines of the one
given in Section \ref{sec.chap.fourier.correl},
except that here we are considering random functions,
and there we were considering conventional functions.
We start with the squared magnitude of the Fourier transform
of the windowed function $X_T$:
%
\begin{equation}
\label{eq-ftsq}
| \Xfo_T | ^ 2 = \Xfo_T \Xfo_T ^ *
\end{equation}
%
Recall that the Fourier transform of a convolution is the product of the
Fourier transforms;
and, since $X_T$ is real
%
\begin{equation}
\label{eq-ftconj}
\Xfo^*_T (f)
= \intone X_T (t) e ^ {2 \pi i ft} \, dt 
= \intone X_T (-t) e ^ {-2 \pi i ft} \, dt 
= \Fo [ X_T (-t) ]
\end{equation}
%
Combining the convolution theorem with (\ref{eq-ftconj})
and (\ref{eq-ftsq}), we have
%
\begin{equation}
\label{eq-ftsq2}
| \Xfo_T | ^ 2 = \Fo [ X_T (t) \conv X_T (-t) ]
= \Fo \left [ \intone X_T (s) X_T (s - t) \, ds \right ]
\end{equation}
%
In (\ref{eq-psddef1}) we have normalized by the interval $T$,
so we put that
into the definition of another function:
%
\begin{equation}
\label{eq-autofrmpsd}
R_T (t) = \frac{1}{T} X_T (t) \conv X_T (-t)
= \frac{1}{T} \intone X_T (s) X_T (s - t) \, ds 
\end{equation}
%
Then by (\ref{eq-ftsq2}), the Fourier transform of $R_T$ is
%
\begin{equation}
\label{eq-ftsq3}
\Fo [ R_T ] = \intone e ^ {-2 \pi i ft} R_T (t) \, dt
= \frac{1}{T}\Fo [ X_T (t) \conv X_T (-t) ] =
\frac{| \Xfo_T | ^ 2}{T}
\end{equation}
%
Our definition of the PSD is (\ref{eq-psddef1});
let us plug (\ref{eq-ftsq3}) into that so that we get
%
\begin{equation}
\label{eq-psdftsq}
S(f) = 
\lim_{T \to \infty} \Ex \left [
\frac{| \Xfo_T | ^ 2}{T} \right ] = 
\lim_{T \to \infty} 
\intone e ^ {-2 \pi i ft } \Ex [ R_T (t) ] \, dt
\end{equation}
%
From (\ref{eq-autofrmpsd}) we see that $R_T (t)$ is even in $t$,
so we can always write $R_T (t) = R_T ( | t | )$;
in the following we will assume 
$t \ge 0$ and then replace $t$ by $| t |$ at the end.
We know $X_T (s)$
vanishes outside the interval $(-T/2, T/2)$ and therefore the integrand
of (\ref{eq-autofrmpsd}) must vanish when $ s > T/2$ or when $| s - t | > T/2$;
see Figure \ref{fig.chap.psd.integrand}.
Therefore we can reduce the interval of integration in
(\ref{eq-autofrmpsd}) to be over $(-T/2 + t, \,T/2)$ instead of the whole real line.
Also observe that, once $t > T/2$,
the nonzero sections cease to overlap,
and the integrand is identically zero.

These considerations lead to the result
%
\begin{equation}
\label{eq-autofin}
R_T (t) = 
\left \{
\begin{array}{cc}
\frac{1}{T} \int_{-T/2+ t }^{T/2} X_T (s) X_T (s - t ) \, ds & 0 \le t < T/2 \\
0 & t \ge T/2 \\
\end{array}
\right .
\end{equation}
%
Further simplifications ensue when we take the expected value, as
dictated by (\ref{eq-psdftsq}); for the segment $0 \le t < T/2$
%
\[
\Ex [ R_T (t) ] = \frac{1}{T} \int_{-T/2+ t }^{T/2}
\Ex [ X_T (s) X_T (s - t ) ] \, ds
= \frac{1}{T} \int_{-T/2+ t }^{T/2} R(-t) \, ds
\]
%
where we have, finally,
introduced the actual autocovariance of the process, $R(t)$.
Since $R(-t) = R(t)$, which is independent
of $s$, we can evaluate the $s$ integral explicitly:
%
\begin{equation}
\Ex [ R_T (t) ] = \frac{R(t)}{T} \int_{-T/2+ t }^{T/2}
1 \cdot \, ds = R(t) \left [ 1 - \frac{t}{T} \right ]
\qquad \mathrm{for} \quad 0 \le t \le T
\end{equation}
%
From (\ref{eq-autofin}) $\Ex [ R_T (t) ] = 0$ when $t \ge T$.
Recalling that $R_T$
is even, we can write
the negative $t$ behavior from $R_T (t) = R_T (-t)$,
and obtain the following complete description for the
expected value of $R_T$:
%
\begin{equation}
\label{eq-expauto}
\Ex [ R_T (t)] = R(t) \Lambda_T (t)
\end{equation}
%
where
%
\[
\Lambda_T (t) =
\left \{
\begin{array}{cc}
1 - | t | / T & |t| \le T \\
0 & | t | > T \\
\end{array}
\right \}
\]
%
which is just a triangle function of unit height
and width $2T$.

Substituting (\ref{eq-expauto}) into
(\ref{eq-psdftsq}) gives us the following
very plausible expression for the PSD:
%
\begin{equation}
\label{eq-psdfrmauto}
S(f) = 
\lim_{T \to \infty} 
\Fo [ R(t) \Lambda_T (t) ] = 
\lim_{T \to \infty} 
\intone e ^ {-2 \pi i ft } R(t) \Lambda_T (t) \, dt
\end{equation}

If we can put the limit in (\ref{eq-psdfrmauto}) inside the integral,
and since $ \Lambda_T (s) \to 1$ as $T \to \infty$,
we then have the result we claimed:
equation (\ref{eq-psddef2}).
This last step is where some care is needed.
\citet{Priestley81} (pp. ??)
uses the Lebesgue Dominated Convergence Theorem, and the
further condition that
%
\begin{equation}
\label{eq-autorestr}
\intone | R(t) | \, dt < \infty 
\end{equation}
%
to prove that it is permitted to reverse the order of the limit 
and the integral.
Section \ref{sec.chap.psd.proof} contains
a proof (by Parker)
that makes a different set of assumptions about $R(t)$.


%%
%%
%%
\section{A from-scratch example: White noise.}
A straightforward way to understand normalization in spectral analysis
is to analyze a real, stationary series which is normally distributed with
known variance, $x=\mathcal{N}(\mu,\sigma^2)$.
A fundamental result found in many texts on spectral analysis is
%
\begin{equation}
\mathrm{var}\{x\} \equiv \sigma_x^2 
= \int_{-1/2}^{1/2} {\dPSD{}} (f) df 
= 2 \int_{0}^{1/2} {\dPSD{}} (f) df 
\end{equation}
%
which says if we integrate the \idx{power spectral density}
over all frequencies we can obtain the variance of the source process.
If we have a $\mathcal{N}(0, 1)$ process, and assume the sampling
interval is once per second, we should expect a
 flat spectrum of 2 units$^2/$Nyquist 
across all frequencies $[0, 0.5]$ so that the area under the spectrum is
equal to one.

We can illustrate this with a few lines of code.
First, generate a series, and then find its iscrete Fourier Transform (DFT)\footnote{
A proper DFT is normalized by the length of the series; however, most
DFT calculators (including \Rcmd{stats::fft}) eschew this normalization for 
efficiency's sake.
}.
\begin{knitrout}
\definecolor{shadecolor}{rgb}{0.969, 0.969, 0.969}\color{fgcolor}\begin{kframe}
\begin{alltt}
\hlfunctioncall{set.seed}(1234)
N <- 256
x <- \hlfunctioncall{rnorm}(N, mean = 0, sd = 1)
xv <- \hlfunctioncall{var}(x)
X <- \hlfunctioncall{fft}(x)
\hlfunctioncall{class}(X)
\end{alltt}
\begin{verbatim}
## [1] "complex"
\end{verbatim}
\begin{alltt}
\hlfunctioncall{length}(X)
\end{alltt}
\begin{verbatim}
## [1] 256
\end{verbatim}
\end{kframe}
\end{knitrout}


We can easily find the amplitude and phase response:
\begin{knitrout}
\definecolor{shadecolor}{rgb}{0.969, 0.969, 0.969}\color{fgcolor}\begin{kframe}
\begin{alltt}
Sa <- \hlfunctioncall{Mod}(X)  \hlcomment{# Amplitude spectrum}
Sp <- \hlfunctioncall{Arg}(X)  \hlcomment{# Phase spectrum}
\end{alltt}
\end{kframe}
\end{knitrout}

followed by equivalent \idx{energy spectral density}
calculations\footnote{
Note the equivalence
between the complex conjugate based estimates.
}
\begin{knitrout}
\definecolor{shadecolor}{rgb}{0.969, 0.969, 0.969}\color{fgcolor}\begin{kframe}
\begin{alltt}
XC <- \hlfunctioncall{Conj}(X)
\hlfunctioncall{all.equal}(Se <- Sa^2, Se_2 <- \hlfunctioncall{Mod}(XC * X), Se_2R <- \hlfunctioncall{Mod}(X * XC))
\end{alltt}
\begin{verbatim}
## [1] TRUE
\end{verbatim}
\end{kframe}
\end{knitrout}


The single-sided \idx{power spectral density} (PSD) estimates
follow once the Nyquist frequency is set; this
is defined as half the sampling rate\footnote{
Although a white noise process is not strictly bandlimited,
we will use it to demonstrate differences in normalization.
}.
\begin{knitrout}
\definecolor{shadecolor}{rgb}{0.969, 0.969, 0.969}\color{fgcolor}\begin{kframe}
\begin{alltt}
fsamp <- 1  \hlcomment{# sampling freq, Hz}
fNyq <- fsamp/2  \hlcomment{# nyquist}
Nf <- N/2
nyfreqs <- \hlfunctioncall{seq.int}(from = 0, to = fNyq, length.out = Nf)
S <- Se[1:Nf] * 2/N  \hlcomment{# Finally, the PSD!}
\end{alltt}
\end{kframe}
\end{knitrout}


\begin{figure}[htb!]
\begin{center}
\begin{knitrout}
\definecolor{shadecolor}{rgb}{0.969, 0.969, 0.969}\color{fgcolor}\begin{kframe}
\begin{alltt}
\hlfunctioncall{plot}(nyfreqs, S, type = \hlstring{"h"}, xlab = \hlstring{"Nyquist frequency"}, ylab = \hlstring{"units**2 / freq"})
\hlfunctioncall{print}(\hlfunctioncall{c}(mSn <- \hlfunctioncall{mean}(S), mSm <- \hlfunctioncall{median}(S)))
\end{alltt}
\begin{verbatim}
## [1] 2.034 1.242
\end{verbatim}
\begin{alltt}
\hlfunctioncall{abline}(h = \hlfunctioncall{c}(mSn, mSm), lwd = 2, lty = \hlfunctioncall{c}(2, 3), col = \hlstring{"red"})
\end{alltt}
\end{kframe}
\includegraphics[width=\maxwidth]{figure/PSD} 

\end{knitrout}

%polygon(c(0,fNyq,fNyq,0,0), c(0,0,mSn,mSn,0), lwd=2, lty=2, border="red")
\label{fig:psdN}
\caption{Power spectral density estimates for a single realization of a 
$\mathcal{N}(0,1)$ process in linear units.  
The dashed line shows the mean spectral level and the dotted line
shows the median spectral level; these can be
used to find the integrated spectrum and test normalization.}
\end{center}
\end{figure}

An estimate of the integrated spectrum
should roughly equal the known variance.
Figure \ref{fig:psdN} plots the PSD of our white noise series; it also shows
the mean value of the PSD\footnote{
Estimates for the PSD of a white noise series
are approximately log-normally distributed; hence,
a simple mean value is highly biased estimator.
}, from which we can perform a variance--normalization
test:
\begin{knitrout}
\definecolor{shadecolor}{rgb}{0.969, 0.969, 0.969}\color{fgcolor}\begin{kframe}
\begin{alltt}
test_norm <- \hlfunctioncall{function}(sval, nyq, xvar) \{
    svar <- sval * nyq
    \hlfunctioncall{return}(svar/xvar)
\}
\hlfunctioncall{print}(xv_1 <- \hlfunctioncall{test_norm}(mSn, fNyq, xv))
\end{alltt}
\begin{verbatim}
## [1] 0.9933
\end{verbatim}
\begin{alltt}
xv_2 <- \hlfunctioncall{sum}(S)/Nf * fNyq/xv  \hlcomment{# an alternate test}
\hlfunctioncall{all.equal}(xv_1, xv_2)
\end{alltt}
\begin{verbatim}
## [1] TRUE
\end{verbatim}
\end{kframe}
\end{knitrout}


But what if the sampling frequency \texttt{fsamp} changes? An obvious change will be
the actual Nyquist frequency, which means the variance--normalization test will
fail if the PSD estimates are not re-scaled.  We simply re-scale the frequencies
and PSD
with the sampling rate
to obtain the properly-normalized spectra.

\begin{knitrout}
\definecolor{shadecolor}{rgb}{0.969, 0.969, 0.969}\color{fgcolor}\begin{kframe}
\begin{alltt}
fsamp <- 20
fNyq <- fsamp/2
freqs <- fsamp * nyfreqs
Snew <- S/fsamp
\end{alltt}
\end{kframe}
\end{knitrout}


To compare the scalings it is helpful to instead show the spectral values
in decibels (relative to 1 units$^2/$frequency).
\begin{figure}[htb!]
\begin{center}
\begin{knitrout}
\definecolor{shadecolor}{rgb}{0.969, 0.969, 0.969}\color{fgcolor}\begin{kframe}
\begin{alltt}
\hlcomment{# decibel function}
dB <- \hlfunctioncall{function}(y) 10 * \hlfunctioncall{log10}(y)
\hlcomment{# and some plots...}
\hlfunctioncall{plot}(freqs, \hlfunctioncall{dB}(S), type = \hlstring{"h"}, xlab = \hlstring{"Frequency"}, ylab = \hlstring{"dB"})
\hlfunctioncall{lines}(freqs, \hlfunctioncall{dB}(Snew), col = \hlstring{"blue"}, lwd = 2)
\hlfunctioncall{abline}(h = \hlfunctioncall{dB}(1/fNyq), col = \hlstring{"grey"}, lwd = 2)
mSn <- \hlfunctioncall{mean}(Snew)
\hlfunctioncall{lines}(\hlfunctioncall{c}(0, fNyq), \hlfunctioncall{rep}(\hlfunctioncall{dB}(mSn), 2), lwd = 2, lty = 2, col = \hlstring{"red"})
\end{alltt}
\end{kframe}
\includegraphics[width=\maxwidth]{figure/PSD2} 
\begin{kframe}\begin{alltt}
\hlcomment{# finally, test variance.}
\hlfunctioncall{test_norm}(mSn, fNyq, xv)
\end{alltt}
\begin{verbatim}
## [1] 0.9933
\end{verbatim}
\end{kframe}
\end{knitrout}

\label{fig:psdsamp}
\caption{Rescaled PSD estimates for a single realization of a 
$\mathcal{N}(0,1)$ process with a sampling rate of 20 s$^{-1}$ rather
than 1 s$^{-1}$ as from before.  
The dashed line shows the mean (rescaled) spectral level, and the
grey line shows the predicted mean value from the Nyquist
frequency.}
\end{center}
\end{figure}

\section{Normalization used in \Rcmd{stats::spectrum}}
We wish to compare the normalizations used by other PSD estimation programs;
these are summarized in \ref{tbl:methods}.
% needs
%\newcommand{\rlp}[0]{\Rcmd{rlpSpec}}
%\newcommand{\SC}[1]{\textsc{#1}}
%\newcommand{\SCY}[0]{\SC{Yes}}
%\newcommand{\SCN}[0]{\SC{No}}
\begin{table}[htbp!]
\centering
\caption{A comparison of functions comparable to \rlp{}, excluding
extensions which only estimate raw-periodograms.}
\begin{tabular}{r l c c l}
\hline
\SC{Function} & \SC{Namespace} & \SC{Sine m.t.?} & \SC{Adaptive?} & \SC{Reference} \\
\hline
\Rcmd{mtapspec} & \Rcmd{RSEIS} & \SCY{} & \SCN{} & \citet{lees1995} \\
\Rcmd{spectrum} & \Rcmd{stats} & \SCN{} & \SCN{} & \citet{rcore} \\
\Rcmd{spec.mtm} & \Rcmd{multitaper} & \SCY{} & \SCY{} & \citet{rahim2012} \\
\Rcmd{SDF} & \Rcmd{sapa} & \SCY{} & \SCN{} & \citet{percival1993} \\
\hline
\end{tabular}
\label{tbl:methods}
\end{table}


\subsection{}
Included in the core distribution of R is \Rcmd{stats::spectrum}, which
accesses \Rcmd{stats::spec.ar} or \Rcmd{stats::spec.pgram} for either
parametric and non-parametric estimation, respectively.  
The user can optionally apply a single cosine taper, and/or a smoothing kernel.
For this discussion we compare to \Rcmd{spec.pgram}.

\Rcmd{spec.pgram} assumes the sampling frequency
for the input series is 1, and normalizes accordingly; however,
sampling information used be included by creating a \Rcmd{ts}
object from the series prior to spectrum estimation:

\begin{knitrout}
\definecolor{shadecolor}{rgb}{0.969, 0.969, 0.969}\color{fgcolor}\begin{kframe}
\begin{alltt}
fsamp <- 20
xt <- \hlfunctioncall{ts}(x, frequency = fsamp)
pgram20 <- \hlfunctioncall{spec.pgram}(xt, pad = 1, taper = 0, plot = FALSE)
pgram01 <- \hlfunctioncall{spec.pgram}(\hlfunctioncall{ts}(xt, frequency = 1), pad = 1, taper = 0, plot = FALSE)
\end{alltt}
\end{kframe}
\end{knitrout}


A first order question is obviously whether these spectra pass our
 variance-normalization test: they do not, but only by a factor of two (too small):

\begin{knitrout}
\definecolor{shadecolor}{rgb}{0.969, 0.969, 0.969}\color{fgcolor}\begin{kframe}
\begin{alltt}
\hlfunctioncall{test_norm}(\hlfunctioncall{mean}(pgram01$spec), 0.5, xv)
\end{alltt}
\begin{verbatim}
## [1] 0.4845
\end{verbatim}
\begin{alltt}
\hlfunctioncall{test_norm}(\hlfunctioncall{mean}(pgram20$spec), 10, xv)
\end{alltt}
\begin{verbatim}
## [1] 0.4845
\end{verbatim}
\end{kframe}
\end{knitrout}


\begin{figure}[h!]
\begin{center}
\begin{knitrout}
\definecolor{shadecolor}{rgb}{0.969, 0.969, 0.969}\color{fgcolor}\begin{kframe}
\begin{alltt}
\hlfunctioncall{plot}(pgram20, log = \hlstring{"dB"}, ylim = 36 * \hlfunctioncall{c}(-1, 0.3))
\hlfunctioncall{plot}(pgram01, log = \hlstring{"dB"}, add = TRUE, col = \hlstring{"red"})
\hlfunctioncall{abline}(h = \hlfunctioncall{dB}(\hlfunctioncall{c}(1, 1/2/1, 1/2/20)), col = \hlfunctioncall{c}(\hlstring{"grey"}, \hlstring{"red"}, \hlstring{"black"}))
\end{alltt}
\end{kframe}
\includegraphics[width=\maxwidth]{figure/NORMS} 

\end{knitrout}

\label{fig:pgram}
\caption{\Rcmd{spec.pgram}}
\end{center}
\end{figure}

But why?  The program assumes normalization for a 
\bidx{double-sided spectrum}, which conflicts with our definition of the 
\idx{one-sided} spectrum
by a factor of two.
We can illustrate this with the following example:

\begin{knitrout}
\definecolor{shadecolor}{rgb}{0.969, 0.969, 0.969}\color{fgcolor}\begin{kframe}
\begin{alltt}
psd1 <- \hlfunctioncall{spec.pgram}(x, plot = FALSE)
psd2 <- \hlfunctioncall{spec.pgram}(xc <- \hlfunctioncall{complex}(real = x, imag = x), plot = FALSE, demean = TRUE)
mx <- \hlfunctioncall{mean}(\hlfunctioncall{Mod}(x))
mxc <- \hlfunctioncall{mean}(\hlfunctioncall{Mod}(xc))
(mxc/mx)^2
\end{alltt}
\begin{verbatim}
## [1] 2
\end{verbatim}
\begin{alltt}
\hlfunctioncall{mean}((psd2$spec/psd1$spec))
\end{alltt}
\begin{verbatim}
## [1] 2
\end{verbatim}
\end{kframe}
\end{knitrout}


This means that unless we are interested in analyzing complex
timeseries, we need only multiply by two for properly normalized spectra 
using \Rcmd{spectrum}, assuming the sampling information is included in the series.

\section{Other PSD estimators}
The suite of extensions to base R which have
similar functionality
is relatively limited; however, there are at least three with
can produce sophisticated PSD estimates.   We have
summarized the available functions in Table \ref{tbl:methods}
so far as we know\footnote{
As of this writing (Feb 2013), \Rcmd{sapa} appears to be orphaned.
}.

% needs
%\newcommand{\rlp}[0]{\Rcmd{rlpSpec}}
%\newcommand{\SC}[1]{\textsc{#1}}
%\newcommand{\SCY}[0]{\SC{Yes}}
%\newcommand{\SCN}[0]{\SC{No}}
\begin{table}[htbp!]
\centering
\caption{A comparison of functions comparable to \rlp{}, excluding
extensions which only estimate raw-periodograms.}
\begin{tabular}{r l c c l}
\hline
\SC{Function} & \SC{Namespace} & \SC{Sine m.t.?} & \SC{Adaptive?} & \SC{Reference} \\
\hline
\Rcmd{mtapspec} & \Rcmd{RSEIS} & \SCY{} & \SCN{} & \citet{lees1995} \\
\Rcmd{spectrum} & \Rcmd{stats} & \SCN{} & \SCN{} & \citet{rcore} \\
\Rcmd{spec.mtm} & \Rcmd{multitaper} & \SCY{} & \SCY{} & \citet{rahim2012} \\
\Rcmd{SDF} & \Rcmd{sapa} & \SCY{} & \SCN{} & \citet{percival1993} \\
\hline
\end{tabular}
\label{tbl:methods}
\end{table}


%% bib
\bibliographystyle{apalike}
\bibliography{REFS}

\printindex
\end{document}
